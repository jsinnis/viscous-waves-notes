\documentclass{article}
\usepackage[utf8]{inputenc}
 \usepackage{enumerate}
\usepackage{amsmath,amsthm,amssymb,amsfonts}
\usepackage{array}
\usepackage{pmgraph}
\usepackage[normalem]{ulem}
\usepackage{fancyhdr}
\usepackage{graphicx}
\usepackage{caption}
\usepackage{float}
\pagestyle{fancy}
\usepackage{xcolor}
\usepackage{amsmath,amsthm,amssymb,amsfonts, enumitem, fancyhdr, color, comment, graphicx, environ}


\usepackage{tikz}
\usepackage{physics}
\newcolumntype{C}[1]{>{\centering\arraybackslash}m{#1}}
\def\firstcircle{(90:1.75cm) circle (2.5cm)}
  \def\secondcircle{(210:1.75cm) circle (2.5cm)}
  \def\thirdcircle{(330:1.75cm) circle (2.5cm)}
\usepackage{booktabs} % for "\midrule" macro
\usepackage{lipsum} % for filler text
\pagestyle{fancy}

\newcommand{\NN}{\mathbb{N}}              % Natural numbers
\newcommand{\QQ}{\mathbb{Q}}              % Rationals
\newcommand{\RR}{\mathbb{R}}              % Reals
\newcommand{\ZZ}{\mathbb{Z}}              % Integers
\newcommand{\powerset}{\mathcal{P}}       % power set
\newcommand{\setcomp}[1]{\overline{#1}}   % set complement
\newcommand{\negate}{{\sim}}              % negation (logic)
\newcommand{\xor}{\oplus}                 % xor
\newcommand{\N}{\mathbb{N}}
\newcommand{\Z}{\mathbb{Z}}

\pagestyle{fancy}


%%%%%%%%%%%%%%%%%%%%%%%%%%%%%%%%%%%%%%%%%%
\title{Viscous water waves}
\author{James Sinnis}
\date{April 2019}

\begin{document}

\maketitle

\section{Derivation of surface height and velocity potential functions}

The PDE is $$\nabla ^2\phi = 0$$
with boundary conditions

$$(i) \indent \frac{\partial{\eta}}{\partial{t}} - \frac{\partial{\phi}}{\partial{z}} - 2\nu \frac{\partial^2{\eta}}{\partial{x^2}} = 0$$

$$(ii) \indent \pdv{\phi}{t} + g\eta -2\nu \pdv[2]{\phi}{x}=0$$

$$(iii)\indent \pdv{\phi}{z} = 0 \text{\indent as z} \xrightarrow{} -\infty$$

\noindent with (i) and (ii) at the surface (z=0).

\\\vspace{2mm} We recognize that the z and x dependence of $\phi$ is separable, and we postulate that $\eta$ takes the form of a propagating cosine:

$$\eta(x,t) = A\cos{(kx - \omega t)} = A\cos{\theta}$$

This leads to $\phi$ taking the form:

$$\phi(x,z,t) = e^{kz} (C\sin{\theta} + D\cos{\theta})$$

where $\theta = kx - \omega t$. Using $(ii)$ we can find the values of $C$ and $D$ in terms of the initial amplitude $A$:

$$C = \frac{g\omega}{\omega^2+4\nu^2k^4} A$$
$$D = -\frac{2\nu^2k^2g}{\omega^2+4\nu^2k^4} A$$
\end{document}
